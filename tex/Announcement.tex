\documentclass[11pt, oneside]{article}
\usepackage{float}
\usepackage{graphicx}
\usepackage{wrapfig}
\graphicspath{ {image/} }



%% Packages

%% Standard packages
\usepackage{geometry}
\geometry{letterpaper}
\usepackage{changepage}   % for the adjustwidth environment

%% AMS
\usepackage{amssymb}
\usepackage{amsmath}

\usepackage{amscd}

\usepackage{fancyvrb} %% for inline verbatim


%% Commands

\newcommand{\code}[1]{{\tt #1}}
\newcommand{\ellie}[1]{\href{#1}{Link to Ellie}}
% \newcommand{\image}[3]{\includegraphics[width=3cm]{#1}}

%% width=4truein,keepaspectratio]


\newcommand{\imagecentercaptioned}[3]{
   \medskip
   \begin{figure}[htp]
   \centering
    \includegraphics[width=#2]{#1}
    \vglue0pt
    \caption{#3}
    \end{figure}
    \medskip
}

\newcommand{\imagecenter}[2]{
   \medskip
   \begin{figure}[htp]
   \centering
    \includegraphics[width=#2]{#1}
    \vglue0pt
    \end{figure}
    \medskip
}

\newcommand{\imagefloat}[4]{
    \begin{wrapfigure}{#4}{#2}
    \includegraphics[width=#2]{#1}
    \caption{#3}
    \end{wrapfigure}
}


\newcommand{\imagefloatright}[3]{
    \begin{wrapfigure}{R}{0.30\textwidth}
    \includegraphics[width=0.30\textwidth]{#1}
    \caption{#2}
    \end{wrapfigure}
}

\newcommand{\hide}[1]{}


\newcommand{\imagefloatleft}[3]{
    \begin{wrapfigure}{L}{0.3-\textwidth}
    \includegraphics[width=0.30\textwidth]{#1}
    \caption{#2}
    \end{wrapfigure}
}
% Font style
\newcommand{\italic}[1]{{\sl #1}}
\newcommand{\strong}[1]{{\bf #1}}
\newcommand{\strike}[1]{\st{#1}}

% Scripta
\newcommand{\ilink}[2]{\href{{https://scripta.io/s/#1}}{#2}}

% Color
\newcommand{\red}[1]{\textcolor{red}{#1}}
\newcommand{\blue}[1]{\textcolor{blue}{#1}}
\newcommand{\violet}[1]{\textcolor{violet}{#1}}
\newcommand{\highlight}[1]{\hl{#1}}
\newcommand{\note}[2]{\textcolor{blue}{#1}{\hl{#1}}}

% WTF?
\newcommand{\remote}[1]{\textcolor{red}{#1}}
\newcommand{\local}[1]{\textcolor{blue}{#1}}

% Unclassified
\newcommand{\subheading}[1]{{\bf #1}\par}
\newcommand{\term}[1]{{\sl #1}}
\newcommand{\termx}[1]{}
\newcommand{\comment}[1]{}
\newcommand{\innertableofcontents}{}


% Special character
\newcommand{\dollarSign}[0]{{\$}}
\newcommand{\backTick}[0]{\`{}}

%% Theorems
\newtheorem{remark}{Remark}
\newtheorem{theorem}{Theorem}
\newtheorem{axiom}{Axiom}
\newtheorem{lemma}{Lemma}
\newtheorem{proposition}{Proposition}
\newtheorem{corollary}{Corollary}
\newtheorem{definition}{Definition}
\newtheorem{example}{Example}
\newtheorem{exercise}{Exercise}
\newtheorem{problem}{Problem}
\newtheorem{exercises}{Exercises}
\newcommand{\bs}[1]{$\backslash$#1}
\newcommand{\texarg}[1]{\{#1\}}


%% Environments
\renewenvironment{quotation}
  {\begin{adjustwidth}{2cm}{} \footnotesize}
  {\end{adjustwidth}}

\def\changemargin#1#2{\list{}{\rightmargin#2\leftmargin#1}\item[]}
\let\endchangemargin=\endlist

\renewenvironment{indent}
  {\begin{adjustwidth}{0.75cm}{}}
  {\end{adjustwidth}}


%% NEWCOMMAND

% \definecolor{mypink1}{rgb}{0.858, 0.188, 0.478}
% \definecolor{mypink2}{RGB}{219, 48, 122}
\newcommand{\fontRGB}[4]{
    \definecolor{mycolor}{RGB}{#1, #2, #3}
    \textcolor{mycolor}{#4}
    }

\newcommand{\highlightRGB}[4]{
    \definecolor{mycolor}{RGB}{#1, #2, #3}
    \sethlcolor{mycolor}
    \hl{#4}
     \sethlcolor{yellow}
    }

\newcommand{\gray}[2]{
\definecolor{mygray}{gray}{#1}
\textcolor{mygray}{#2}
}

\newcommand{\white}[1]{\gray{1}[#1]}
\newcommand{\medgray}[1]{\gray{0.5}[#1]}
\newcommand{\black}[1]{\gray{0}[#1]}

% Spacing
\parindent0pt
\parskip5pt

\begin{document}

\title{Announcement}

\date{}

\author{}

\maketitle

\maketitle


\setcounter{section}{0}


\tableofcontents



\imagecentercaptioned{thmb-7xn0oIF6a9eJ-y_4OO5vN0lJhCg-1500x0-filtersno_upscalemax_bytes150000strip_icc-humming-bird-flowers-GettyImages-1271839175-b515cb4f06a34e66b084ba617995f00a.eps}{0.51\textwidth,keepaspectratio}{Humming bird}

\subsection{About Scripta} \label{about-scripta}

\begin{itemize}

\item{This is a demo of the \u{Scripta Markup Language}.   Compare source and rendered text to see how it works. Your document is rendered as you type.   There is no setup ... just have at it.}

\item{You can't save documents right now, but you will be able to do that as soon as the full scripta app is released.}

\item{Use the megaphone icon on the right to report bugs, ask questions, and make suggestions.   This an early alpha release of Scripta, so you \textbf{will} find bugs. We love to hear about them.}

\item{Note the use of our experimental   \u{ergonomic TeX}: TeX without backslashes.}

\item{Press ctrl-E to export your file to LaTeX.   This feature does not yet work with images, so for now you will have to hide them with a \u{hide} or
\u{code} block.   A fix is on its way}

\end{itemize}

\subsection{Examples} \label{examples}

\newcommand{\secpder}[2]{\frac{\partial^2 #1}{ \partial #2^2}}
\newcommand{\nat}{\mathbb{N}}
\newcommand{\reals}{\mathbb{R}}
\newcommand{\pder}[2]{\frac{\partial #1}{ \partial #2}}
\newcommand{\set}[1]{\{ #1 \}}
\newcommand{\sett}[2]{\{ #1 \ | \ #2 \}}

Pythagoras said: $a^2 + b^2 = c^2$.

This will be on the test:

\begin{equation}
\int_0^1 x^n dx = \frac{1}{n+1}
\end{equation}

and so will this:

\begin{equation}
\label{wave-equation}\secpder{u}{x} + \secpder{u}{y} + \secpder{u}{z} = \frac{1}{c^2} \secpder{u}{t}
\end{equation}

Both of the above equalities were written using an \verb`equation` block.   If you look
at the source text you will see that \eqref{wave-equation} an \u{argument} \verb`numbered` and
a property, namely   \verb`label:wave-equation`. That property is used for cross-referencing: we say \verb`[eqref wave-equation]` to make a hot link to \eqref{wave-equation}.   Click on it now
to see what happens.

Here is an \u{aligned} block:

\begin{align}
\nat &= \set{\text{positive whole numbers and zero}}\nat &= \sett{n \text{ is a whole number}}{ n > 0}
\end{align}

\begin{equation}
\begin{pmatrix}
2 & 1 \\
1 & 2
\end{pmatrix}
\begin{pmatrix}
2 & 1 \\
1 & 2
\end{pmatrix}
=
\begin{pmatrix}
5 & 4 \\
4 & 5
\end{pmatrix}
\end{equation}

\end{document}
